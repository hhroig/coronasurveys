\documentclass{article}
\usepackage{hyperref}
\usepackage{graphicx}
\begin{document}
\title{Measuring Icebergs: Using Different Methods to Estimate the Number of COVID-19 Cases in Portugal and Spain}
\author{$@$CoronaSurveys Team \\ \texttt{https://github.com/GCGImdea/coronasurveys}\\(draft in progress, not peer-reviewed)}
\maketitle
\section{Introduction}

During the current coronavirus pandemic, monitoring the evolution of COVID-19 cases is very important for the authorities to make
informed policy decisions, and for the general public that has the right to be informed of the reach of the problem. 
Official numbers of confirmed cases are periodically issued by each country's health authority. (For Portugal see \url{https://covid19.min-saude.pt}.)
Unfortunately, upon the pandemic outbreak is it usually the case that there is lack of available laboratory tests, and other material and human resources. Hence, it is not possible to test all potential cases, and some eligibility criteria is applied to decide who is tested.
Under these circumstances, the evolution of official confirmed cases might not represent the total number of cases (see \url{https://www.nature.com/articles/d41586-020-00760-8}).  

This motivates the need of other probing techniques, beyond laboratory testing, that may bring in information about the potential numbers of cases. In fact, it is very likely that the true numbers might only be known in the future, once serological surveys are done (as done in prior outbreaks \url{https://journals.plos.org/plosone/article?id=10.1371/journal.pone.0050770}. However, any rough estimate that is within one order of magnitude of the real number will be better than being almost in the dark.

In the rest of this document we propose a new approach to estimate the number of cases with COVID-19 symptoms, which is based on using crowdsourcing with open anonymous surveys to obtain indirect information. We compare this new approach with approaches that infer current cases from the case fatality series for two particular countries, Portugal and Spain.

\section{Delay-adjusted Case Fatality Ratio}

If we consider the current number of fatalities and divide it by the current number of confirmed cases, we obtain a naive case fatality ratio (CFR). However, this metric is not using the actual number of cases with known outcomes, since recent cases will still evolve into fatalities and recoveries. By estimating the true number of cases with known outcomes, it is possible to obtain a corrected case fatality ratio (cCFR) that takes into account the average delay from symptoms to death, as in \url{https://journals.plos.org/plosone/article?id=10.1371/journal.pone.0006852}.
Since the corrected denominator (cases with known outcomes) is reduced, the cCFR is higher than the naive CFR during a growing outbreak. A high cCFR is typically an indicator of lack of coverage in laboratory testing. If we assume, in general, that the disease will have similar case fatality rates in different countries, it is possible to use the known fatality ratio from Wuhan, China, (currently at $1.38\%$). This value is compared with the obtained cCFR, and the percentage of coverage (the proportion of cases that are, in fact, confirmed) in different countries can be estimated. This is done in \url{https://cmmid.github.io/topics/covid19/severity/global_cfr_estimates.html} and, for instance, the projected coverage in Portugal and Spain is $19\%$ and $4.7\%$, respectively. These values can in turn be used to correct the number of reported confirmed cases in each country, and estimate the likely true number of cases, thus probing the iceberg that lurks under water.

Another technique, based on the same corrections for delay, but using an overall gross value for estimating the true cases from the mortality rate is simply obtained by multiplying the cumulative mortality by 400. See \url{https://www.nature.com/articles/d41586-020-00760-8}. We will use both techniques for estimation in each country.

\section{Crowdsourcing with Open Anonymous Surveys}

The $@$CoronaSurveys effort is based on crowdsourcing data collection, in a way that avoids querying the citizens about their particular health status and identity. Participants answer for a given geographical area, which can be a whole country or a region within a country. They answer two simple questions: 

\begin{itemize}
\item How many people do you know in this geographical area? (Please, consider only people whose current health status you likely know.)
\item As far as you know, how many of the above have symptoms compatible with COVID-19 (or were diagnosed with the disease)?
\end{itemize}

By not asking any personal information we aim to protect users privacy, and having just two questions aims to simplify the answering process. This will hopefully increase the participation. However, the lack of detailed information of the participants makes the estimation process challenging. We do not control the spread of the survey nor the adequate coverage of regions, age groups and other parameters. Somewhat suprisingly, even given these limitations, we can still obtain a rough estimate and see that it is not far from those obtained with other techniques. One reason for this may be that each participant is typically reporting on the health status of a large sample (hundreds), which increases significantly the coverage of the survey. Obvious advantages of this approach are that it is very simple to deploy and can give very timely results.

The process to obtain the estimate of cases is as follows.
Survey responses are cleaned by identifying and removing outliers. These are answers unusually large in terms of the range of persons that the respondent declares to know (we remove entries outside 1.5 times the interquartile range above the upper quartile), and the ratio of symptomatic people reported (we remove entries above $30\%$ of reporting). The latter are removed because we aim at surveying the general population that is not in particular high contact with symptomatic cases. Once the data is clean, we use it to obtain an estimate of the percentage of cases in the population that are known to the respondents. Then, we naively extrapolate this ratio to the whole country population.

Our initial surveys were done on Twitter and did not ask for how many people are known to the respondents. For those, the number of known people was set to 150, which is the Dunbar number (\url{https://en.wikipedia.org/wiki/Dunbar%27s_number}), the expected number of persons with whom one maintains stable social relationships. 

In the following sections we show the results we obtain for the two countries, Portugal and Spain. 

\section{The Portuguese data}

Portuguese data on fatality and confirmed cases was obtained in \url{https://github.com/dssg-pt/covid19pt-data}, a public repository that helps disseminated the official DGS daily data.

\begin{figure}
\begin{center}
\includegraphics[width=.9\linewidth]{EstPTMar28.pdf}
\end{center}
\caption{Case estimates for Portugal, March 28th 2020.}
\label{pt}
\end{figure}

In Figure~\ref{pt} the solid line represents how many confirmed cases have been reported. The dashed line is the fatality number multiplied by 400, while the solid dots $\bullet$ indicate the more precise estimate from the corrected case fatality rates and compensating for the estimated coverage of testing. Finally, the diamond $\diamond$ represents our estimates from the two initial Twitter surveys and the triangles $\triangle$ the results from the subsequent two surveys via Google Forms. 

As we can see the anonymous surveys, by $@$CoronaSurveys, tend to estimate more cases, likely over-estimating, but still follow the overall tendency and even might be closer to the order of magnitude of the true numbers than the official reports. 
%On March 26, in Portugal, the official cases were 4268, the cCFR was reporting 25662 and our survey indicated 69839.  

\section{The Spanish data}

Spanish data was obtained from the European Center for Disease Control in \url{https://www.ecdc.europa.eu/}, with public information about COVID-19 cases across the World. 

\begin{figure}
\begin{center}
\includegraphics[width=.9\linewidth]{EstSPMar28.pdf}
\end{center}
\caption{Case estimates for Spain, March 28th 2020.}
\label{sp}
\end{figure}

For the Spanish case, see Figure~\ref{sp}, there is a even wider gap between cases predicted by fatality numbers and the reported cases. We include data points from one twitter survey and three subsequent Google Forms surveys. Surprisingly, the results from $@$CoronaSurveys are very close to the fatality based estimators, while providing still slightly higher estimates. 

\section{Discussion}

In this first report we draw attention to the limitations of relying only on confirmed cases to measure the true size of a growing pandemic. From existing data it is possible to derive other measures, and, surprisingly, it is also possible to do simple surveying approaches that, while simple and maybe crude, are clearly non invasive of privacy, and still get meaningful data. In particular  $@$CoronaSurveys can be relevant in countries with a decent digital infrastructure but lacking in laboratory resources. When measuring icebergs,  there are many strategies. 

\end{document}
