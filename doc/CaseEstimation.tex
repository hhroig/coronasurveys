\documentclass{article}
\usepackage{hyperref}
\begin{document}
\title{Measuring Icebergs: Different Estimates of COVID-19 Cases in Portugal}
\author{$@$CoronaSurveys Team \\ \texttt{https://github.com/GCGImdea/coronasurveys}}
\maketitle
\section{Introduction}

During the current coronavirus pandemics, monitoring the evolution of COVID-19 cases is an important component for awareness and to inform policy decisions. 
Official numbers of laboratory confirmed cases are periodically issued by each country's health authority, for Portugal see \url{https://covid19.min-saude.pt}. 

Typically, along an outbreak testing eligibility criteria can evolve and the number of available tests also. Under these circumstances the evolution of official laboratory confirmed cases might not represent the total number of cases (see \url{https://www.nature.com/articles/d41586-020-00760-8}).  

This motivates looking at other probing techniques and try to bring in more information about the potential numbers. Actually the true numbers might only be known in the future once serological surveys are done (as done in prior outbreaks \url{https://journals.plos.org/plosone/article?id=10.1371/journal.pone.0050770}.

Bellow we consider two approaches: (a) Inferring current cases from the case fatality series; (b) Using crowdsourcing with anonymous surveys of indirect information.  

\section{Delay-adjusted case fatality ratio}

\section{Crowdsourcing with anonymous surveys}

\section{The Portuguese data}

\section{The Spanish data}


\end{document}
